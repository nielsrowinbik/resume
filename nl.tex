\documentclass[10pt]{article}
\usepackage{cv}
\usepackage[dutch]{babel}
\usepackage[]{csquotes}

\cvname{Niels Bik}
\cvpersonalinfo{
    \mailto{hey@nielsbik.nl} | \href{https://www.linkedin.com/in/nielsrowinbik}{linkedin.com/in/nielsrowinbik} | \href{https://www.github.com/nielsrowinbik}{github.com/nielsrowinbik}
}
\begin{document}

\makecvtitle

\section{Werkervaring}
\label{sec:work}

\cvsubsection{Stuvia}[Product Manager][Remote][2023-heden]
\begin{itemize}
    \item Verantwoordelijk voor het formaliseren van product management praktijken en het groeien van het product als Stuvia's eerste product-werknemer.
    \item Een productvisie en -strategie geïntroduceerd in lijn met de bedrijfsstrategie om acquisitie in nieuwe markten en retentie in bestaande markten te verbeteren, gefocust op het verbeteren van de eerst indruk en de (mobiele) UI en ervaring.
    \item Het toevoegen van een interne UX-designer geleid, and de gewoonte geïntroduceerd om regelmatig echte gebruikers te interviewen.
    \item Een set performante landingspagina's gebouwd met Next.js.
\end{itemize}

\cvsubsection{ProRail}[Product Manager][Utrecht][2018-2022]
\begin{itemize}
    \item Vier product teams aangestuurd en fundamentele Scrum-waarden geherintroduceerd in het ontwikkelproces van \enquote{Donna} (software gebruikt voor het toekennen van capaciteit voor alle treinen op het Nederlandse spoornet, met de dienstregeling als belangrijkste output), waaronder de nadruk op het opleveren van werkende software, met een afname van het aantal verstoringen direct na een driemaandelijkse release van 100\% als gevolg.
    \item Voor het eerst sinds de lancering van Donna (2005) een productvisie en -strategie neergezet. Hiermee stakeholder acceptatie verkregen voor de volledige modernisering van het product en op strategisch niveau een verschuiving van denken in “output” naar denken in “outcome” in gang gezet.
    \item Een proces geïntroduceerd om op gestructureerde wijze problemen en hun oplossingen te identificeren en valideren (\enquote{Discovery}). Hiermee de verschuiving van “output” naar “outcome” voortgezet op tactisch en operationeel niveau.
    \item Team van zes product managers aangestuurd dat ontstaan is na het opschalen van de ontwikkelingen aan Donna.
    \item Verbeterde (80+\% minder vals positieven, onbekend aantal minder vals negatieven, 50\% sneller) module voor het signaleren van fouten in de dienstregeling met bijhorende visualisatie gelanceerd, die de kwaliteit van de dienstregeling sterk verbeterd heeft en de fundering legt voor belangrijke technologische vooruitgangen in de spoorsector als Automatic Train Operation.
\end{itemize}

\cvsubsection{ProRail}[Assistent Product Manager (o.a.) (parttime)][Utrecht][2016-2018]
\begin{itemize}
     \item Als eerste lid van de \enquote{ProRail ICT Studentenpool} de werkprocessen vormgegeven, nieuwe collega's ingewerkt, en de bekendheid van de pool vergroot o.a. door een presentatie te organiseren en geven voor alle ICT managers van ProRail.
    \item De product manager van de \enquote{Btd-planner} (software gebruikt voor het toekennen van capaciteit voor alle onderhoudswerkzaamheden op het Nederlandse spoornet) geholpen de controle over zijn Product Backlog terug te krijgen.
\end{itemize}

\section{Opleidingen}
\label{sec:school}

\cvsubsection{Universiteit Utrecht}[Master Business Informatics, entrepreneurial track][Utrecht][2016-2018]
\begin{itemize}
    \item De proof-of-concept webapplicatie \enquote{openSEA} gemaakt, horende bij een scriptie getiteld \enquote{Developing a model-driven socio-environmental auditing tool}, waarin ik een meer open aanpak voor sociale en milieu-gerelateerde audits bepleit.
    \item Auteur van een paper waarin ik een referentiemethode voor het toepassen van User Stories voorstel en coauteur van een paper gebaseerd op het onderzoek in mijn scriptie (zie \hyperref[sec:pubs]{publicaties}).
\end{itemize}

\cvsubsection{Universiteit Utrecht}[Bachelor Informatiekunde][Utrecht][2012-2016]
\begin{itemize}
    \item Als stagiair bij Planbition (\link{www.planbition.eu}) onderzoek gedaan naar technieken om vereisten aan software te ontdekken (\textit{requirements elicitation techniques}) en bevindingen toegepast om eisen aan een nieuwe feature in Planbition's software vast te stellen. Voorgaande gedocumenteerd in een scriptie getiteld \enquote{Requirements specification of automated planning systems for employment agencies}.
\end{itemize}

\section{Publicaties}
\label{sec:pubs}

\begin{itemize}
    \item S. España, N. Bik and S. Overbeek, \enquote{Model-driven engineering support for social and environmental accounting,} \textit{2019 13th International Conference on Research Challenges in Information Science (RCIS)}, 2019, pp. 1-12, doi:\\ 10.1109/RCIS.2019.8877042.
    \item N. Bik, G. Lucassen and S. Brinkkemper, \enquote{A Reference Method for User Story Requirements in Agile Systems Development,} \textit{2017 IEEE 25th International Requirements Engineering Conference Workshops (REW)}, 2017, pp. 292-298, doi: 10.1109/REW.2017.83.
\end{itemize}

\section{Aanvullende informatie en technische kennis}
\label{sec:other}

\begin{itemize}
    \item \textbf{\underline{Visitter}}: Freemium SaaS oplossing voor het beheren van de beschikbaarheid van vakantiehuisjes. Oorspronkelijk gemaakt voor ouders' vakantiehuis, later publiek beschikbaar gemaakt op  \link{https://visitter.app}. Gemaakt met Next.js, PlanetScale, en SendGrid.
    \item \textbf{\underline{Qoffeetime}}: Koffie timer webapplicatie, gemaakt vanuit een persoonlijke behoefte om koffierecepten te timen en om Next.js te leren. Beschikbaar op \link{https://qoffeetime.app}.
    \item Veel ervaring met React en gerelateerde frameworks, libraries en technologieën als Next.js, Tailwind CSS, Webpack, en Git. Beperkte ervaring met andere web frameworks als Vue, Svelte, en Lit.
    \item Basiskennis van Docker en de onderliggende containertechnologie, opgedaan door het zelf hosten van een aantal services.
\end{itemize}

\end{document}
