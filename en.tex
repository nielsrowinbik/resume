\documentclass[10pt]{article}
\usepackage{cv}
\usepackage[english]{babel}
\usepackage{hyperref}
\usepackage{csquotes}

\cvname{Niels Bik}

\cvpersonalinfo{
  \mailto{hey@nielsbik.nl} | \href{https://www.linkedin.com/in/nielsrowinbik}{linkedin.com/in/nielsrowinbik} | \href{https://www.github.com/nielsrowinbik}{github.com/nielsrowinbik}
}

% \hypersetup{final}

\begin{document}

\makecvtitle

\section{Work experience}
\label{sec:work}

\cvsubsection{ProRail}[Product Manager][Utrecht, NL][2018-present]
\begin{itemize}
      \item Led four product teams and revamped the engineering process for
            \enquote{Donna} (software used for capacity allocation of the Dutch railroad
            system, with its primary output being the timetable), re-introducing
            foundational Scrum practices and a focus on delivering working software,
            resulting in 100\% fewer service interruptions directly following each
            quarterly release.
      \item Set product vision and strategy for the first time since Donna's launch
            in 2005. Got stakeholder buy-in for a complete overhaul of the product and
            started a shift at the strategic level from output to outcome.
      \item Established and rolled out processes and tools to enable a structured
            approach to identifying and validating problems and their solutions
            (\enquote{Discovery}), continuing the shift from output to outcome at the
            tactical and operational level.
      \item Led a team of six product managers after scaling up development efforts
            for Donna.
      \item Launched an improved (80+\% fewer false positives, unknown amount of
            fewer false negatives, 50\% faster) conflict-detection module for Donna,
            resulting in a reduced workload for end-users, and forming the foundation for
            Driver Advisory Systems (DAS) and Automatic Train Operation (ATO), important
            technological advancements for the Dutch rail sector.
\end{itemize}

\cvsubsection{ProRail}[Student Worker (parttime)][Utrecht, NL][2016-2018]
\begin{itemize}
      \item Established a way of working and onboarding procedure for the
            \enquote{ProRail ICT student pool}, for which I was the first hire. Increased
            publicity of the pool by arranging and giving a presentation for all managers
            in the ICT business unit.
      \item Helped the product manager for ProRail's \enquote{Btd-planner} (used for
            capacity allocation of 20.000 yearly maintenance activities on the Dutch
            railroad system) take back control of his Product Backlog as an Assistant
            Product Manager.
\end{itemize}

\cvsubsection{Studentaanhuis.nl}[\enquote{Student} (parttime)][Amersfoort and
      Utrecht, NL][2013-2016]
\begin{itemize}
      \item Helped Studentaanhuis.nl customers with computer and technology related
            issues such as WiFi problems, malware-infested computers, and making new
            purchases, during which the main focus was on customer sastisfaction.
\end{itemize}

\section{Education}
\label{sec:school}

\cvsubsection{Utrecht University}[Master in Business Informatics,
      entrepreneurial track][Utrecht, NL][2016-2018]
\begin{itemize}
      \item Built a proof-of-concept web application called \enquote{openSEA}
            accompanying a thesis titled \enquote{Developing a model-driven
                  socio-environmental auditing tool}, building the case for a more open approach
            to socio-environmental auditing.
      \item Authored and published a paper in which I propose a reference method for
            applying User Stories and co-authored and published a paper based on the
            research in my thesis (see \hyperref[sec:pubs]{publications}).
\end{itemize}

\cvsubsection{Utrecht University}[Bachelor's programme in Information
      Science][Utrecht, NL][2012-2016]
\begin{itemize}
      \item Internship at Planbition (\link{www.planbition.eu}, Nijmegen, NL), during
            which I researched requirements elicitation techniques and applied the findings
            to investigate and identify which requirements a new feature in Planbition's
            software should meet. Produced a thesis titled \enquote{Requirements
                  specification of automated planning systems for employment agencies}.
\end{itemize}

\cvsubsection{'t Atrium}[(T)VWO (bilingual pre-university secondary
      education)][Amersfoort, NL][2006-2012]
\begin{itemize}
      \item Stayed in host family in Ireland and participated in exchange programme
            with Canadian peers, gaining invaluable experience in speaking English.
      \item Received the International Baccalaureate diploma for English A2 level
            (near-native).
\end{itemize}

\section{Publications}
\label{sec:pubs}

\begin{itemize}
      \item S. España, N. Bik and S. Overbeek, \enquote{Model-driven engineering
                  support for social and environmental accounting,} \textit{2019 13th
                  International Conference on Research Challenges in Information Science (RCIS)},
            2019, pp. 1-12, doi:\\ 10.1109/RCIS.2019.8877042.
      \item N. Bik, G. Lucassen and S. Brinkkemper, \enquote{A Reference Method for
                  User Story Requirements in Agile Systems Development,} \textit{2017 IEEE 25th
                  International Requirements Engineering Conference Workshops (REW)}, 2017, pp.
            292-298, doi: 10.1109/REW.2017.83.
\end{itemize}

\section{Additional information and technical skills}
\label{sec:other}

\begin{itemize}
      \item \textbf{\underline{Qoffeetime}}: Built and published a coffee timer web
            app out of a personal need to time coffee brewing recipies and the desire to
            learn Next.js. Available at \link{https://qoffeetime.app}.
      \item Extensive experience with React and related frameworks, libraries and
            technologies such as Next.js, Tailwind CSS, Webpack, and Git. Limited
            experience with other web frameworks such as Vue, Svelte, and Lit.
      \item Basic knowledge of Docker, having used it to self-host several services,
            and underlying container technology.
\end{itemize}

\end{document}